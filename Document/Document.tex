\documentclass[11pt,a4paper]{article}
\usepackage[italian]{babel}
\usepackage[utf8]{inputenc}
\usepackage{graphicx}
\usepackage{xcolor}
\graphicspath{{Loghi}}
\usepackage{tabularx}
\usepackage[colorlinks=true, urlcolor=blue, linkcolor=black, citecolor=black]{hyperref}
\usepackage[none]{hyphenat}
\usepackage{enumitem}
\usepackage{fancyhdr}
\usepackage{geometry}
\geometry{margin=2cm}
\usepackage{tikz}

% Campo compilabile + riga tratteggiata stampabile SOTTO il bordo inferiore
% #1 = opzioni per \TextField (es. name=Docente)
% #2 = testo (di solito vuoto: {})
\newcommand{\PrintableField}[2][]{%
	\begin{tikzpicture}[remember picture,baseline=(field.base)]
		% Nodo "field" che contiene il TextField
		\node[inner sep=0pt,outer sep=0pt,anchor=base west] (field)
		{\TextField[#1,width=\linewidth]{#2}};
		% Riga tratteggiata sul bordo inferiore del campo
		\draw[densely dotted]
		(field.south west) -- (field.south east);
	\end{tikzpicture}%
}

% Campo con riga stampabile sotto, a larghezza fissa
% #1 = opzioni (es. name=LiData)
% #2 = larghezza (es. 3cm)
\newcommand{\FixedPrintableField}[2]{%
	\TextField[#1,width=#2]{}% campo PDF
	\\[-0.3ex]% piccolissimo spazio verso l'alto
	\rule{#2}{0.4pt}% riga continua di stessa larghezza
}

% Campo a larghezza fissa con riga sotto, ma senza alterare il layout:
% in modalità "schermo" mostriamo solo il TextField (come fai già),
% in stampa aggiungiamo la riga.
\newcommand{\PrintLine}[1]{%
	% #1 = larghezza (es. 3cm)
	\leavevmode\rule{#1}{0.4pt}%
}

\begin{document}
	
	% Remove page number
	\pagestyle{empty}
	
	\begin{center}
		\Large
		\textbf{Modulo giustificativo assenza dal lavoro per partecipare alla seduta d'esame}
	\end{center}
	
	\vspace*{1cm}
	
	\begin{tabularx}{\linewidth}{@{}lX@{}}
		\textbf{Docente} &
		\PrintableField[name=Docente]{}
	\end{tabularx}
	
	\vspace*{0.2cm}
	
	\begin{tabularx}{\linewidth}{@{}lX@{}}
		\textbf{Insegnamento} &
		\PrintableField[name=Insegnamento]{}\newline
		\PrintableField[name=Insegnamento1]{}
	\end{tabularx}
	
	\vspace*{0.2cm}
	
	\begin{tabularx}{\linewidth}{@{}lX@{}}
		\textbf{Si dichiara che lo studente} &
		\PrintableField[name=Studente]{}\newline
		\PrintableField[name=Studente1]{}
	\end{tabularx}
	
	\vspace*{0.2cm}
	
	\begin{tabularx}{\linewidth}{@{}lX@{}}
		\textbf{Matricola} &
		\PrintableField[name=Matricola]{}
	\end{tabularx}
	
	\vspace*{0.2cm}
	
	\begin{tabularx}{\linewidth}{@{}lX@{}}
		\textbf{iscritto al Corso di Laurea in} &
		\PrintableField[name=Corso]{}
	\end{tabularx}
	
	\vspace*{0.2cm}
	
	\begin{tabularx}{\linewidth}{@{}lX@{}}
		\textbf{ha sostenuto in data odierna l’esame di} &
		\PrintableField[name=Esame]{}\newline
		\PrintableField[name=Esame1]{}\newline
		\PrintableField[name=Esame2]{}
	\end{tabularx}
	
	\vspace*{0.2cm}
	
	\begin{tabularx}{\linewidth}{@{}lX@{}}
		\textbf{presso la sede} &
		\PrintableField[name=Sede]{}\newline
		\PrintableField[name=Sede1]{}
	\end{tabularx}
	
	\vspace*{0.2cm}
	
	\begin{tabularx}{\linewidth}{@{}lX@{}}
		\textbf{altro} &
		\PrintableField[name=Altro]{}\newline
		\PrintableField[name=Altro1]{}
	\end{tabularx}
	
	\vspace*{1cm}
	
	% riga dei campi "lì"
	\TextField[name=liData,width=3cm]{} lì \TextField[name=liLuogo,width=5cm]{}
	\vspace*{-0.45cm}
	
	% riga stampabile sotto (stesso layout)
	\PrintLine{3cm}\hspace{1.3em}\PrintLine{5cm}
	
	\vspace*{1cm}
	
	\begin{minipage}{\linewidth}
		\textbf{N.B. Il presente modulo deve essere compilato a cura dello studente e convalidato da un
			docente del Corso di Laurea di appartenenza, per gli usi consentiti.}
	\end{minipage}
	
\end{document}